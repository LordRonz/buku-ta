\chapter{PENUTUP}
\label{chap:penutup}

% Ubah bagian-bagian berikut dengan isi dari penutup

\section{Kesimpulan}
\label{sec:kesimpulan}

Berdasarkan hasil pengujian yang mulai dari pengujian sistem Sharing Data Audio berbasis blockchain pada Smart Contract hingga integrasi unreal engine 5, diperoleh beberapa kesimpulan sebagai berikut:

\begin{enumerate}[nolistsep]

  \item \emph{Smart contract} yang dibuat membutuhkan rerata konsumsi gas sebesar 3275354 ether untuk melakukan \emph{deployment smart contract} di \emph{testnet}, dan 3274368 di \emph{Ganache}. Untuk pemanggilan \emph{method mint} dibutuhkan \emph{gas} sebesar 125574 di \emph{testnet}

  \item Sistem ini memiliki beberapa ketentuan format yang mengikat mengingat pada pengujian diperlukan data yang valid pada \emph{JSON metadata}, yang memiliki setidaknya 1 \emph{key} \texttt{audio\_url}.

  \item Sinergi untuk web3 tercapai dengan adanya integrasi dengan smart contract ethereum yang juga interoperable dengan blockchain lain seperti solana dan polygon.

  \item Terdapat 3 file audio yang dapat dibaca oleh Unreal Engine 5 secara real time, yakni mp3, flac, dan wav.

\end{enumerate}

\section{Saran}
\label{chap:saran}

Untuk pengembangan lebih lanjut pada penelitian ini, terdapat beberapa saran yang bisa dilakukan khususnya pada sistem data sharing berbasis blockchain ini, yang mana diantara lain:

\begin{enumerate}[nolistsep]

  \item Menggunakan \emph{metasound} pada versi Unreal Engine 5.3 yang akan datang karena akan semakin erat kaitannya dengan web3 dan metaverse. Pada penelitian ini menggunakan audio player bawaan dari Unreal Engine 5.

  \item Membuat sistem validasi yang lebih \emph{robust} sehingga dapat mencegah error runtime baik pada smart contract dan juga Unreal Engine 5 terkhusus pada bagian blueprint.

  \item Membuat support yang lebih pada beberapa format audio maupun format metadata yang lebih fleksibel sehingga mempermudah baik pengembang maupun pengguna.

\end{enumerate}
