\chapter{METODOLOGI}
\label{chap:desainimplementasi}

% Ubah bagian-bagian berikut dengan isi dari desain dan implementasi

Penelitian ini dilaksanakan sesuai desain sistem berikut beserta implementasinya. 
Desain sistem merupakan konsep dari pembuatan dan perancangan infrastruktur dan kemudian 
diwujudkan dalam bentuk blok-blok alur yang harus dikerjakan. Pada bagian implementasi merupakan 
pelaksanaan teknis untuk setiap blok pada desain sistem.

\section{Deskripsi Sistem}
\label{sec:deskripsisistem}

Sistem akan dibuat dengan \lipsum[1-2]

\section{Implementasi Alat
\label{sec:implementasi alat}}

Alat diimplementasikan dengan \lipsum[1]

% Contoh pembuatan potongan kode
\begin{lstlisting}[
  language=C++,
  caption={Program halo dunia.},
  label={lst:halodunia}
]
#include <iostream>

int main() {
    std::cout << "Halo Dunia!";
    return 0;
}
\end{lstlisting}

\lipsum[2-3]

% Contoh input potongan kode dari file
\lstinputlisting[
  language=Python,
  caption={Program perhitungan bilangan prima.},
  label={lst:bilanganprima}
]{program/bilangan-prima.py}

\lipsum[4]
