\chapter{PENDAHULUAN}
\label{chap:pendahuluan}

% Ubah bagian-bagian berikut dengan isi dari pendahuluan

\section{Latar Belakang}
\label{sec:latarbelakang}

Munculnya teknologi \emph{virtual reality (VR)} dan pengembangan platform metaverse telah mengarah pada terciptanya
pengalaman online yang imersif di mana pengguna dapat berinteraksi satu sama lain dan terlibat dalam berbagai aktivitas, apalagi setelah  Mark Zuckerberg
membeli \emph{oculus} \parencite{luckerson2014facebook}.
Salah satu aktivitas tersebut adalah memainkan alat musik sebagai karakter virtual, yang dapat dicapai melalui penggunaan
teknologi \emph{motion capture} dan simulasi instrumen virtual.

Di era saat ini, blockchain sangat penting bagi industri musik dan metaverse karena menyediakan solusi untuk beberapa masalah yang dihadapi kedua industri tersebut.
Blockchain dapat membantu industri musik menangani masalah seperti keaslian, transparansi, dan pengelolaan royalti. Informasi kepemilikan dan hak cipta musik dapat dilacak secara terbuka dan tidak dapat diubah dengan teknologi blockchain. Ini menjaga kekayaan intelektual para pencipta dan memastikan bahwa royalti yang seharusnya mereka terima didistribusikan dengan adil. Selain itu, blockchain memungkinkan pencipta musik untuk menjual musik mereka secara langsung kepada penggemar mereka dengan menggunakan NFT (Non-Fungible Token), yang menawarkan pengalaman interaktif dan keuntungan baru dalam hal monetisasi.

Metaverse sangat penting untuk menciptakan pengalaman digital yang lebih kaya, inklusif, dan terhubung. Metaverse membawa potensi baru dalam bidang hiburan, interaksi sosial, ekonomi digital, pendidikan, dan banyak lagi. Dengan perkembangan teknologi dan adopsi metaverse yang semakin luas, kita dapat mengantisipasi perubahan besar dalam cara kita berinteraksi dan mengalami dunia digital.

Kemudian untuk layanan-layanan pemutar musik juga memiliki beberapa implementasi system \emph{centralized}. Contohnya spotify, soundcloud, dan lain-lain.
Soundcloud sendiri memiliki kelemahan pada sisi kepemilikan atau \emph{copyright}. Sistem \emph{decentralized} merupakan salah satu solusi dari hal ini, dikarenakan
sifatnya yang \emph{immutable}. Salah satu contohnya adalah \emph{Audius}. Audius sendiri menggunakan sistem \emph{decentralized} dan source musiknya disimpan di IPFS.
Akan tetapi untuk menggunakan Audius harus menggunakan \emph{embedded iframe} yang artinya hanya support untuk penggunaan di \emph{Web Application} saja.

Adapula antarmuka dari audio itu sendiri adalah metaverse yang salah satu contohnya adalah Unreal Engine 5. Salah satu fitur dari Unreal Engine 5
yaitu Metasound yang dapat melakukan pengolahan audio dengan mudah bagi para pengembang. Unreal Engine 5 juga
memiliki fitur yakni API \emph{blockchain} sehingga dapat mengeksekusi \emph{smart contract}. Fitur ini dinamakan Web3.UE. Web3.Unreal adalah plugin
\emph{open source} yang dibuat untuk pengembang game dan komunitas untuk membantu mereka yang bekerja dengan Unreal Engine untuk mengintegrasikan
fungsionalitas \emph{blockchain} ke dalam game mereka. Kedua hal ini dapat dikombinasikan dan digunakan untuk
pengembangan metaverse.

\emph{Metaverse} telah digambarkan sebagai iterasi baru dari internet yang menggunakan headset VR, teknologi \emph{blockchain},
dan avatar dalam integrasi baru dunia fisik dan virtual. \parencite{DWIVEDI2022102542}. Dalam proposal penelitian ini, diusulkan
untuk merancang dan mengimplementasikan sistem berbagi data berbasis \emph{blockchain}
untuk \emph{musical player} di \emph{metaverse} dengan \emph{NFT}. Sistem ini akan menggunakan \emph{smart contract} dan solusi penyimpanan terdesentralisasi
untuk memungkinkan pengguna berbagi dan mengakses data musik dari \emph{metaverse} dengan cara yang aman dan transparan.

\section{Permasalahan}
\label{sec:permasalahan}

Di dunia \emph{virtual metaverse}, kebutuhan akan sistem yang aman dan efisien untuk berbagi data terkait musik semakin meningkat.
Saat ini, data ini sering disimpan dalam \emph{database} terpusat yang rentan terhadap pelanggaran keamanan dan
penyensoran, dan dapat menyulitkan pengguna untuk mengakses dan mengontrol data mereka sendiri. Ketidakmampuan berinteroperabilitas 
dalam ekosistem \emph{blockchain} dapat menyebabkan fragmentasi data dan aset, keterbatasan fungsionalitas dan kasus penggunaan, penurunan skalabilitas, 
likuiditas terbatas, fragmentasi pasar, serta kurangnya kolaborasi dan standardisasi. Interoperabilitas sangat penting untuk mewujudkan potensi penuh teknologi 
\emph{blockchain} dan memungkinkan pertukaran nilai yang lancar di berbagai jaringan. Karena belum ada sistem yang ada untuk melakukan
hal serupa layaknya \emph{audius} dan \emph{spotify} di \emph{metaverse},
maka penelitian ini juga mencetuskan hal yang belum ada pada saat proposal ini diajukan.

\section{Tujuan}
\label{sec:Tujuan}

Penelitian ini memiliki tujuan untuk membuat sistem yang
dapat melakukan sharing data untuk audio player yang ada
di \emph{Metaverse} menggunakan platform \emph{blockchain}, agar
pengguna \emph{metaverse} dapat menggunakan platform music
yang diintegrasikan dengan \emph{metasound}. Penelitian ini juga bertujuan mengembangkan \emph{metaverse} yang
bersinergi dengan web3.0, dan diharapkan terjadinya \emph{interoperability} sehingga tidak hanya dapat digunakan
dengan \emph{Ethereum} saja.

\section{Batasan Masalah}
\label{sec:batasanmasalah}

Untuk memfokuskan permasalahan yang diangkat maka dilakukan pembatasan ma-
salah. Batasan - batasan masalah tersebut diantaranya:
\begin{enumerate}
  \item Jenis \emph{Blockchain} yang digunakan adalah \emph{Ethereum}.
  \item Jaringan \emph{blockchain} yang digunakan adalah \emph{Goerli} dan \emph{Sepolia}
  \item Target pengguna adalah pengguna \emph{metaverse}.
  \item Target platform musik adalah \emph{Metasounds} di Unreal Engine 5.
  \item Data berisi \emph{JSON} yang adalah \emph{metadata} yang diperuntukkan untuk \emph{metasound}.
  \item File audio disimpan di IPFS
\end{enumerate}
