\chapter{TINJAUAN PUSTAKA}
\label{chap:tinjauanpustaka}

% Ubah bagian-bagian berikut dengan isi dari tinjauan pustaka

Demi mendukung penelitian ini, \lipsum[1][1-5]

\section{Penelitian Terdahulu}
\label{sec:penelitianterdahulu}

% Contoh input gambar
\subsection{Sistem Transaksi Antar Player Pada Game Multiplayer Wisata Bromo Menggunakan Blockchain}
\label{subsec:sistemtransaksiblockchainreza}
Penelitian oleh Reza Putra Pradana berjudul Sistem Transaksi Antar Player Pada Game Multiplayer Wisata
Bromo merupakan implementasi teknologi Blockchain untuk sistem transaksi antar player dalam bentuk game
dengan mengujikan latensi serta performa waktu eksekusi transaksi hingga biaya gas yang dibutuhkan untuk
pengembangan game tersebut. Dalam penelitian ini Blockchain yang digunakan salah satunya adalah Ethereum untuk pengujiannya.
Adanya biaya gas kaitannya erat dengan proses \emph{development} dari game itu sendiri karena melalui biaya gas
dapat diketahui berapa lama proses transaksi dilakukan sehingga analisis lebih lanjut diperlukan untuk mendapat formula yang pas terkait biaya gas dengan kebutuhan \emph{development} dari game tersebut.

\section{\emph{Blockchain}}
\label{sec:blockchain}
Teknologi blockchain adalah mekanisme database canggih yang memungkinkan berbagi informasi yang transparan di dalam jaringan bisnis.
Basis data blockchain menyimpan data dalam blok yang terhubung satu sama lain dalam sebuah rantai. Data tersebut konsisten secara kronologis
karena tidak dapat dihapus atau dimodifikasi tanpa persetujuan dari jaringan. Sebagai hasilnya, teknologi blockchain dapat digunakan untuk
membuat buku besar tidak dapat diubah atau immutable untuk melacak pesanan, pembayaran, akun, dan transaksi lainnya. Sistem ini memiliki
mekanisme bawaan yang mencegah entri transaksi yang tidak sah dan menciptakan konsistensi dalam pandangan bersama mengenai transaksi-transaksi tersebut.

Untuk mengenal \emph{Blockchain}, dapat dimulai dari pemahaman dari pengertian yang paling bisa diterima,
Menurut Layman, \emph{Blockchain} adalah jaringan yang terus berkembang, aman, dan dibagikan sistem pencatatan
di mana setiap pengguna data memegang salinan catatan, yang hanya dapat diperbarui jika semua pihak yang
terlibat dalam suatu transaksi setuju untuk memperbarui. Sedangkan secara teknisnya, \emph{Blockchain} adalah buku besar
terdistribusi peer-to-peer yang aman secara kriptografis, hanya ditambahkan, tidak dapat diubah (sangat sulit untuk perubahan),
dan hanya dapat diperbarui melalui konsensus atau kesepakatan seluruh pihak.

Gravitasi merupakan \lipsum[1]

\section{\emph{Ethereum}}
\emph{Ethereum} adalah jaringan mesin virtual peer-to-peer yang dapat digunakan oleh pengembang mana pun untuk menjalankan aplikasi
terdistribusi (Dapps). Program komputer ini dapat berbentuk apa saja, tetapi jaringan dioptimalkan untuk menjalankan aturan yang
berjalan secara otomatis ketika kondisi tertentu terpenuhi, seperti kontrak. Ethereum menggunakan \emph{Blockchain} publiknya sendiri yang
terdesentralisasi untuk menyimpan, mengeksekusi, dan melindungi kontrak ini secara kriptografis. Setiap komputer di jaringan ini akan
mengunduh mesin virtual kecil untuk disinkronkan dengan \emph{Blockchain Ethereum} dan tetap tersedia untuk menjalankan kontrak.
Jaringan komputer terdistribusi ini dengan mudah menyediakan keamanan, keandalan, dan daya komputasi yang diperlukan untuk melaksanakan
pengaturan yang dirancang. \emph{Blockchain Ethereum} pun juga dapat dicari secara publik

\section{\emph{Smart Contract}}

"\emph{Smart contract}" (kontrak cerdas) adalah program yang disimpan di dalam \emph{blockchain} dan dijalankan ketika
kondisi-kondisi yang telah ditentukan terpenuhi. Umumnya, \emph{smart contract} digunakan untuk mengotomatisasi
pelaksanaan suatu kesepakatan agar semua peserta dapat segera memastikan hasilnya, tanpa ada keterlibatan pihak
perantara atau kehilangan waktu. \emph{Smart contract} juga dapat mengotomatisasi alur kerja dengan memicu tindakan
berikutnya ketika kondisi yang telah ditetapkan terpenuhi.

Kontrak pada dasarnya merupakan cara untuk membentuk kesepakatan persetujuan terhadap suatu hal (Szabo, 1996).
Kontrak secara umum digunakan untuk keperluan dalam membangun protokol persetujuan terhadap suatu hubungan yang dibentuk
oleh dua pihak individu/kelompok atau lebih. Kontrak akan dibangun oleh dua orang atau lebih dengan menggunakan bantuan
supervisi dari pihak ketiga yang dianggap bisa dipercaya. Supervisi ini sangat penting untuk dimiliki untuk menghindari adanya
manipulasi oleh salah satu pihak terhadap kontrak yang dibentuk. Seiring dengan perkembangan teknologi, terbentuk suatu konsep
kontrak baru yang bernama \emph{Smart Contract}.

\section{\emph{Ethereum Transaction}}

Transaksi adalah paket data yang ditandatangani untuk mengirim Ether dari satu akun ke akun lain atau ke kontrak, memanggil
metode kontrak, atau menyebarkan kontrak baru. Sebuah transaksi adalah ditandatangani menggunakan ECDSA (Elliptic Curve Digital
Signature Algorithm), yang merupakan digital algoritma tanda tangan berdasarkan ECC. Sebuah transaksi berisi penerima pesan,
sebuah tanda tangan yang mengidentifikasi pengirim dan membuktikan niat mereka, jumlah Ether yang akan ditransfer, jumlah
maksimum langkah komputasi yang diizinkan untuk dilakukan oleh eksekusi transaksi (disebut batas gas), dan biaya yang
bersedia dibayar oleh pengirim transaksi untuk masing-masing langkah komputasi (disebut harga gas). Jika niat transaksi
adalah untuk memanggil metode kontrak, itu juga berisi data input, atau jika tujuannya adalah untuk menyebarkan kontrak,
maka itu bisa kode inisialisasi. Produk gas yang digunakan dan harga gas disebut transaksi biaya. Untuk mengirim Ether atau
menjalankan metode kontrak, perlu menyalurrkan transaksi ke jaringan. Pengirim perlu menandatangani transaksi dengan kunci
pribadinya (Prusty, 2017)

\section{\emph{InterPlanetary File System}}

IPFS adalah sekumpulan protokol modular untuk mengatur dan mentransfer data, yang dirancang dari awal dengan prinsip-prinsip
addressing konten dan jaringan peer-to-peer. Karena IPFS bersifat open-source, terdapat beberapa implementasi dari IPFS.
Meskipun IPFS memiliki lebih dari satu kasus penggunaan, kasus penggunaan utamanya adalah untuk menerbitkan data
(file, direktori, situs web, dll.) secara terdesentralisasi.

IPFS adalah jaringan terbuka, terdistribusi, dan partisipatif yang mengurangi silo data dari server terpusat, membuat 
IPFS lebih tangguh daripada sistem tradisional. Tidak ada entitas atau individu tunggal yang mengendalikan, mengelola, 
atau memiliki IPFS; sebaliknya, ini adalah proyek yang dijaga oleh komunitas dengan beberapa implementasi dari protokol, 
beberapa alat dan aplikasi yang memanfaatkan protokol tersebut, dan beberapa pengguna dan organisasi yang berkontribusi 
pada desain dan pengembangannya.

IPFS menyediakan akses yang lebih cepat ke data dengan memungkinkannya direplikasi dan diambil dari beberapa lokasi, 
serta memungkinkan pengguna untuk mengakses data dari lokasi terdekat menggunakan addressing konten daripada addressing berbasis lokasi. 
Dengan kata lain, karena data dapat diakses berdasarkan kontennya, sebuah node pada jaringan dapat mengambil data tersebut dari node lain 
di jaringan yang memiliki data; dengan demikian, masalah kinerja seperti laten dapat dikurangi.

\section{\emph{Ethereum Virtual Machine}}

Proses dibalik berjalannya \emph{Smart Contract} pada \emph{Ethereum} diatur dalam suatu mesin komputasi yang diberi nama
\emph{Ethereum Virtual Machine}. Semua penyebaran serta eksekusi dari \emph{Smart Contract} berlangsung dengan melewati mesin virtual ini.
Transaksi transfer nilai sederhana dari satu EOA ke EOA lainnya tidak perlu melibatkannya, tetapi yang lainnya akan melibatkan pembaruan status yang dihitung oleh \emph{EVM}.
Pada tingkat tinggi, EVM yang berjalan di \emph{Blockchain Ethereum} dapat dianggap sebagai komputer terdesentralisasi global yang berisi jutaan objek yang dapat dieksekusi,
masing-masing dengan penyimpanan data permanennya sendiri.

\subsection{Hukum Newton}
\label{subsec:hukumnewton}

Newton \parencite{newton1687} pernah merumuskan bahwa \lipsum[1]
Kemudian menjadi persamaan seperti pada persamaan \ref{eq:hukumpertamanewton}.

% Contoh pembuatan persamaan
\begin{equation}
  \label{eq:hukumpertamanewton}
  \sum \mathbf{F} = 0\; \Leftrightarrow\; \frac{\mathrm{d} \mathbf{v} }{\mathrm{d}t} = 0.
\end{equation}

\subsection{Anti Gravitasi}
\label{subsec:antigravitasi}

Anti gravitasi merupakan \lipsum[1]
