\begin{center}
  \large
  \textbf{LEMBAR PENGESAHAN}
\end{center}

% Menyembunyikan nomor halaman
\thispagestyle{empty}

\begin{center}
  % Ubah kalimat berikut dengan judul tugas akhir
  \textbf{KALKULASI ENERGI PADA ROKET LUAR ANGKASA BERBASIS \emph{ANTI-GRAVITASI}}
\end{center}

\begingroup
% Pemilihan font ukuran small
\small

% \vspace{3ex}

\begin{center}
  \textbf{TUGAS AKHIR}
  \\Diajukan untuk memenuhi salah satu syarat memperoleh gelar Sarjana Teknik pada Program Studi S-1 Teknik Komputer Departemen Teknik Komputer Fakultas Teknologi Elektro dan Informatika Cerdas Institut Teknologi Sepuluh Nopember
\end{center}

% \vspace{3ex}

\begin{center}
  % Ubah kalimat berikut dengan nama dan NRP mahasiswa
  Oleh: \name{}
  \\NRP. \nrp{}
\end{center}

% \vspace{3ex}

% \begin{center}
% Ubah kalimat-kalimat berikut dengan tanggal ujian dan periode wisuda
%   Tanggal Ujian : 1 Juni 2021\\
%   Periode Wisuda : September 2021
% \end{center}

\begin{center}
  Disetujui oleh Tim Penguji Tugas Akhir:
\end{center}

% \vspace{4ex}

\begingroup
% Menghilangkan padding
\setlength{\tabcolsep}{0pt}

\noindent
\begin{tabularx}{\textwidth}{X l}
  % Ubah kalimat-kalimat berikut dengan nama dosen pembimbing pertama
  \advisor{}                       & (Pembimbing I)                      \\
  NIP: \advisornip{}               &                                     \\
                                   & ................................... \\
                                   &                                     \\
                                   &                                     \\
  % Ubah kalimat-kalimat berikut dengan nama dosen pembimbing kedua
  \coadvisor{}                     & (Pembimbing II)                     \\
  NIP: \coadvisornip{}             &                                     \\
                                   & ................................... \\
                                   &                                     \\
                                   &                                     \\
  % Ubah kalimat-kalimat berikut dengan nama dosen penguji pertama
  Dr. Galileo Galilei, S.T., M.Sc. & (Penguji I)                         \\
  NIP: 18560710 194301 1 001       &                                     \\
                                   & ................................... \\
                                   &                                     \\
                                   &                                     \\
  % Ubah kalimat-kalimat berikut dengan nama dosen penguji kedua
  Friedrich Nietzsche, S.T., M.Sc. & (Penguji II)                        \\
  NIP: 18560710 194301 1 001       &                                     \\
                                   & ................................... \\
                                   &                                     \\
                                   &                                     \\
  % Ubah kalimat-kalimat berikut dengan nama dosen penguji ketiga
  Alan Turing, ST., MT.            & (Penguji III)                       \\
  NIP: 18560710 194301 1 001       &                                     \\
                                   & ................................... \\
                                   &                                     \\
                                   &                                     \\
\end{tabularx}
\endgroup

% \vspace{2ex}

\begin{center}
  % Ubah kalimat berikut dengan jabatan kepala departemen
  Mengetahui, \\
  Kepala Departemen \department{} \facultyshort{} - ITS\\

  \vspace{8ex}

  % Ubah kalimat-kalimat berikut dengan nama dan NIP kepala departemen
  \underline{Dr. Supeno Mardi Susiki Nugroho, S.T., M.T.} \\
  NIP. 19700313 199512 1 001
\end{center}

\begin{center}
  \textbf{\MakeUppercase{\place{}}\\Bulan, \the\year{}}
\end{center}
\endgroup
