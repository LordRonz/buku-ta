\begin{center}
  \large\textbf{ABSTRACT}
\end{center}

\addcontentsline{toc}{chapter}{ABSTRACT}

\vspace{2ex}

\begingroup
% Menghilangkan padding
\setlength{\tabcolsep}{0pt}

\noindent
\begin{tabularx}{\textwidth}{l >{\centering}m{3em} X}
  % Ubah kalimat berikut dengan nama mahasiswa
  \emph{Name}     & : & \name{}                                                             \\

  % Ubah kalimat berikut dengan judul tugas akhir dalam Bahasa Inggris
  \emph{Title}    & : & \emph{Anti-Gravity Based Energy Calculation on Outer Space Rockets} \\

  % Ubah kalimat-kalimat berikut dengan nama-nama dosen pembimbing
  \emph{Advisors} & : & 1. \advisor{}                                                       \\
                  &   & 2. \coadvisor{}                                                     \\
\end{tabularx}
\endgroup

% Ubah paragraf berikut dengan abstrak dari tugas akhir dalam Bahasa Inggris
\emph{The Metaverse is a set of virtual spaces, where one can create\\
  and browse with other internet users who are not in the same physical space\\
  with that person. The application often uses blockchain as a solution\\
  decentralized. The Metaverse itself certainly requires an audio output, which is realized\\
  by Unreal Engine 5 with its metasound feature.\\
  \\
  Merging the two can be achieved by using smart contracts. Then in\\
  This research will create a system that can make this happen with the help of Ethereum\\
  Smart Contract, NFT to store metadata, and IPFS web3.storage to store\\
  audio source files for metasound.\\
  \\
  With the existence of a decentralized system, it is also hoped that interoperability will be realized.\\
  ability to make this metaverse usable and communicate with other blockchain platforms.}

% Ubah kata-kata berikut dengan kata kunci dari tugas akhir dalam Bahasa Inggris
\emph{Keywords}: \emph{Rocket}, \emph{Anti-gravity}, \emph{Energy}, \emph{Space}.
